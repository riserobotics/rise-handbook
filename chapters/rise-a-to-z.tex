\section{Regeln und Richtlinien}

\subsection{Einsatz von KI im Projekt und den Pruefungsleistungen}
Im Allgemeinen ist der Einsatz von KI Werkzeugen in diesem Modul nicht verboten, bei der Nutzung sind aber Regeln zur beachten. 

Der Einsatz von KI im Rahmen von Projekten und Prüfungsleistungen ist zulässig, sofern die Eigenleistung der Studierenden eindeutig erkennbar bleibt. Jede Nutzung von KI ist deutlich zu kennzeichnen und in der Selbstständigkeitserklärung unter Angabe der verwendeten Programme, Methoden und Zeitpunkte zu dokumentieren. Zur Transparenz sind Tool, verwendete Prompts, Ausgaben und ggf. Chatverläufe vollständig im digitalen Anhang aufzuführen.

KI ist nicht als Quelle anzusehen; sie kann fehlerhafte Inhalte erzeugen. Alle recherchierten Informationen müssen eigenständig geprüft und mit vollständigen Quellenangaben (inkl. DOI) belegt werden.

Ziel ist die Effizienzsteigerung bei gleichbleibender Qualität durch methodischen Einsatz von KI. Das Fachgebiet überprüft und entwickelt den Umgang mit KI fortlaufend weiter. Anpassungen – insbesondere zwischen Semestern – sind möglich; Vorschläge von Studierenden sind jederzeit willkommen.

\epigraph{"Der Einsatz generativer Modelle im Rahmen des
wissenschaftlichen Arbeitens sollte angesichts der erheblichen
Chancen und Entwicklungspotenziale keinesfalls
ausgeschlossen werden. Ihr Einsatz erfordert jedoch bestimmte
verbindliche Rahmenbedingungen, um die gute
wissenschaftliche Praxis und die Qualität wissenschaftlicher
Ergebnisse zu sichern."}{-- DFG Stellungnahme}

\subsection{Sprache}
Die primäre Sprache im Modul ist Deutsch. Vorlesungen, Workshops und Anleitung werden in der Regel auf Deutsch gehalten, ein Sprachniveau was es erlaubt diesen hinreichend zu folgen ist hilfreich. Es ist jedoch absolut möglich die Module auch auf Englisch zu belegen, Praesentationen und Dokumentation auf Englisch zu halten bzw. zu verfassen. Wir werden soweit es moeglich ist versuchen alle Materialien sowohl auf Deutsch als auch auf Englisch zur Verfuegrung zu stellen, im Zweifel ist jedoch die Deutsche Version bindend. 

In den Projektphasen innerhalb der Gruppen haben wir in der Vergangenheit gute Erfahrungen mit rein englischsprachigen Teams gemacht. Hierzu gibt es aber keine feste Regel, sondern es werden im Rahmen der Gruppenzusammenstellung individuelle Lösungen gefunden. 

Neben den Inhalten des Moduls selbst, z.B. Veranstaltungen oder Präsentationen, ist in den Arbeitsphasen innerhalb der Gruppen auch Interaktion mit Ressourcen außerhalb der Universität erforderlich, z.B. Datenblätter, Dokumentationen von Komponenten oder Software. In sehr vielen Fällen, insbesondere im elektrotechnischen \& Softwarebereich, liegen diese Dokumente nur auf Englisch vor - ein hinreichend gutes Verständnis der englischen Sprache ist also ebenfalls hilfreich. Teilweise ist, z.B. in den RISE Softwareprojekten, auch die interne Dokumentation sowie die Schriftsprache Englisch.  

Die Kommunikation mit dem Modulverantwortlichen und Tutor*innen kann sowohl auf Deutsch als auch auf Englisch stattfinden. Andere Sprachen als Deutsch oder Englisch können wir im Modul nicht unterstützen. 
\newpage