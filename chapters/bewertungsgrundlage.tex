\section{Bewertungsgrundlagen}

% Optional (keine Pakete nötig):
\renewcommand{\arraystretch}{1.2}
\setlength{\tabcolsep}{6pt}

% -------- Tabelle 1: Inhalt (60%) --------
\begin{table}[htbp]
\centering
\caption{Faktoren für die Bewertung der Zwischen- und Endpräsentationen in den RISE-Lehrmodulen — Inhalt}
\vspace{0.2cm}
\begin{tabular}{|p{0.20\textwidth}|p{0.64\textwidth}|p{0.12\textwidth}|}
\hline
\multicolumn{3}{|l|}{\textbf{Inhalt \hfill 60\%}}\\ \hline
\textbf{Teilaspekt} & \textbf{Beschreibung} & \textbf{Anteil}\\ \hline

Zeitplan &
{\raggedright
Die gegebene Aufgabe ist vollständig und sinnvoll in voneinander unterscheidbare Teilaufgaben unterteilt.\par
Die Teilaufgaben sind zeitlich sinnvoll im vorgegebenen Zeitraum angeordnet.\par
Abweichungen sind begründet und Gegenmaßnahmen wurden vorgeschlagen bzw.\ vorgenommen.\par
Der aktuelle Fortschritt ist dem geplanten Fortschritt gegenübergestellt.\par
}
& 15\% \\ \hline

Methodik &
{\raggedright
Zur Lösung der Problemstellung wurde eine geeignete Methodik ausgewählt.\par
Die Durchführung der Methodik ist ausreichend dokumentiert.\par
Abweichungen von der Methodik sind entsprechend gekennzeichnet und begründet.\par
}
& 15\% \\ \hline

Ergebnisse &
{\raggedright
Zwischenstände bzw.\ Ergebnisse sind zielgerichtet und übersichtlich dargestellt.\par
Die dargestellten Parameter dienen der Beantwortung der Fragestellung.\par
}
& 15\% \\ \hline

Interpretation und Diskussion &
{\raggedright
Die Interpretation der Daten erfolgt in angemessenem Maße.\par
Die Grenzen der Aussagekraft gewonnener Erkenntnisse sind ausreichend dargestellt.\par
Offene Fragen bzw.\ Themenstellungen sind angesprochen.\par
Abgeleitete Aussagen sind belastbar, gut begründet und durch die dargestellten Daten unterstützt.\par
}
& 15\% \\ \hline
\end{tabular}
\end{table}

\clearpage % sorgt dafür, dass die erste Tabelle abgeschlossen wird und die nächste auf neuer Seite beginnt

% -------- Tabelle 2: Präsentation (40%) --------
\begin{table}[htbp]
\centering
\caption{Faktoren für die Bewertung der Zwischen- und Endpräsentationen in den RISE-Lehrmodulen — Präsentation}
\vspace{0.2cm}
\begin{tabular}{|p{0.20\textwidth}|p{0.64\textwidth}|p{0.12\textwidth}|}
\hline
\multicolumn{3}{|l|}{\textbf{Präsentation \hfill 40\%}}\\ \hline
\textbf{Teilaspekt} & \textbf{Beschreibung} & \textbf{Anteil}\\ \hline

Strukturierte Darstellung der Inhalte &
{\raggedright
Ein roter Faden ist in der Darstellung der Inhalte erkennbar.\par
Die dargestellten Inhalte sind im Hinblick auf Relevanz gewichtet und wurden entsprechend der Gewichtung ausreichend umfangreich für die Darstellung (z.\,B.\ gewidmete Redezeit, Anzahl der Folien etc.) erhalten.\par
Kausale Ketten bauen in ihrer dargestellten Reihenfolge logisch aufeinander auf.\par
}
& 10\% \\ \hline

Redeanteile &
{\raggedright
Die inhaltlichen Redeanteile der Gruppenmitglieder sind gleich verteilt.\par
Fragen in der Diskussion können von unterschiedlichen Mitgliedern beantwortet werden.\par
}
& 10\% \\ \hline

Rhetorik &
{\raggedright
Die Aussprache der Präsentierenden ist frei und gut verständlich.\par
Formulierungen sind klar, deutlich und grammatisch korrekt wiedergegeben.\par
Die Nutzung von Füllworten ist in angemessenem Maße.\par
}
& 10\% \\ \hline

Visuelle Darstellung &
{\raggedright
Fachbegriffe werden in angemessenem Umfang und sachlich richtig verwendet.\par
Dargestellte Inhalte sind übersichtlich und prägnant.\par
Große Themenkomplexe werden schrittweise vorgestellt.\par
Textbausteine sind aus angemessener Entfernung lesbar.\par
Foliennummern sind vorhanden und gut sichtbar.\par
Präsentationsfolien sind einheitlich formatiert.\par
}
& 10\% \\ \hline

\end{tabular}
\end{table}
